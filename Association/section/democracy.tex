\section{Democratic Process}
\label{sec:democratic-process}
\begin{enumerate}
    \item Wyvern has community meetings, administration meetings, and committee meetings.
    
    \item Administration and committee meetings are managed internally by their respective body.

    \item With exception of article~\ref{sec:democratic-process}.\ref{itm:meeting-exceptional}, community meetings are organized by the Director.
    
    \item \label{itm:meeting-exceptional} If 3 Wyvern members submit a written request indicating the desire for a community meeting, the Director is obligated to organize a community meeting within one week. If the Director refrains, the aforementioned members hold the right to organize a community meeting themselves.

    \item Wyvern decision-making is to take place during community meetings, in which \emph{votes} will be held.
    
    \item The Director leads community meetings. In absence of the Director, the meeting decides by vote whom shall lead the meeting.

    \item Only Wyvern members with voting rights may partake in a vote.
    
    \item Only Wyvern members who pay contribution to the respective term have voting rights.

    \item Before a vote is held, the Director must allow for members to abstain from voting. Abstaining is only possible by leaving the room in which the vote is held. If the meeting is held electronically, abstaining members must leave the call.
    
    \item Wyvern members who are neither physically nor digitally present abstain on all votes by default.

    \item Wyvern members who have a conflict of interest abstain by default. A conflict of interest is present if the respective Wyvern member is the subject of the vote.

    \item A vote consists of multiple options, each of which may be voted for. The Director decides whether members may vote for multiple options.
    
    \item A vote must be preceded by an open discussion in which potential options may be discussed. The Director then decides which options have enough support to be included in the vote before the vote may start. Each vote must include a blank option, indicating no preference.

    \begin{item}
        There are two types of votes:
        \begin{enumerate}
            \item majority votes;
            \item unanimous votes.
        \end{enumerate}
    \end{item}

    \item With exception of articles~\ref{sec:democratic-process}.\ref{itm:vote-money} and~\ref{sec:democratic-process}.\ref{itm:vote-policy}, the Director decides whichever vote type is applicable before the vote is started.

    \item \label{itm:vote-money} If a vote is related to Wyvern funds, it must be unanimous.

    \item \label{itm:vote-policy} If a vote is related to the Wyvern's policy, guidelines, or rules, it must be unanimous.

    \begin{item}
        For a majority vote to succeed, the following conditions must be met:
        \begin{enumerate}
            \item at least 50\% Wyvern members vote (non-abstaining);
            \item at least one Wyvern administrator participates (non-abstaining).
        \end{enumerate}
    \end{item}

    \begin{item}
        For a unanimous vote to succeed, the following conditions must be met:
        \begin{enumerate}
            \item at least two-thirds of Wyvern members vote (non-abstaining);
            \item at least one Wyvern administrator participates (non-abstaining);
            \item only a single option with exception of the blank option has more than zero votes.
        \end{enumerate}
    \end{item}

    \begin{item}
        In a successful vote, the result is decided by the greater number of votes. If the result is a tie, the administration team may (re)discuss the options, after which the Director settles the vote by either:
        \begin{enumerate}
            \item taking an executive decision, thus deciding upon an option;
            \item restarting the vote, possibly with a modified set of options; or
            \item postponing the vote, possibly indefinitely.
        \end{enumerate}
    \end{item}

    \begin{item}
        In an unsuccessful vote, the administration may (re)discuss the options, after which the Director settles the vote by either:
        \begin{enumerate}
            \item restarting the vote, possibly with a modified set of options; or
            \item postponing the vote, possibly indefinitely.
        \end{enumerate}
    \end{item}

\end{enumerate}
