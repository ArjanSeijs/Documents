\section{Guidelines \& Regulations}
\label{sec:bureaucracy}
This section describes the process in which the Wyvern project is governed. In particular, the roles, responsibilities and capabilities of the administration team are formally introduced. In addition, the voting process is further described. Lastly, the financial responsibilities of Wyvern members are introduced.

\subsection{Administration Team Roles}
\begin{enumerate}
    \item The Wyvern project has an administration team consisting of three Wyvern members: the \emph{director}, the \emph{chief of finance}, and the \emph{chief of technology}.
    
    \begin{item}
        The responsibilities and duties of the director are:
        \begin{enumerate}
            \item to represent the interests and desires of all Wyvern members in the Wyvern administration team;
            \item to oversee the administration team, ensuring the administrators are in compliance with Wyvern policies;
            \item to facilitate Wyvern community meetings, ensuring the democratic process is properly adhered to;
            \item to plan and reserve a space for Wyvern community meetings;
            \item to promote good communication and resolve conflicts within the Wyvern project.
        \end{enumerate}
    \end{item}

    \begin{item}
        The responsibilities and duties of the chief of finance are:
        \begin{enumerate}
            \item to ensure the financial obligations of the Wyvern project are met;
            \item to ensure the financial obligations of Wyvern members are met;
            \item to record and update financial transactions and information for the Wyvern server;
            \item to act as secretary during Wyvern meetings, documenting all Wyvern discussions and votings in meetings;
            \item to oversee other financial aspects of the Wyvern project.
        \end{enumerate}
    \end{item}

    \begin{item}
        The responsibilities and duties of the chief of technology are:
        \begin{enumerate}
            \item to ensure the proper functioning of the Wyvern server;
            \item to implement technical changes on the Wyvern server in accordance with the desires of the Wyvern server;
            \item to track and report on changes of Wyvern server during community meetings;
            \item to manage security-related aspects of the Wyvern server, thus ensuring a secure working environment for Wyvern members.
        \end{enumerate}
    \end{item}
\end{enumerate}

\subsection{Democratic Process}
\begin{enumerate}
    \item Wyvern decision-making is to take place during Wyvern meetings, in which \emph{votings} will be held.

    \item Solely the director is in charge of the voting process and may initiate, suspend, resume, and restart a voting.

    \item Before a vote is held, the director must allow for members to abstain from voting. Abstaining is only possible by leaving the room in which the voting is held.
    
    \item Wyvern members whom are neither physically nor digitally present during a meeting abstain on all votes by default.

    \item Only Wyvern members whom contribute money to the applicable half-yearly term may vote. The director decides upon which term the vote is applicable. Potential Wyvern members do not have a vote.
    
    \item A voting must be preceded by an open discussion of at least 5 minutes, where potential options are discussed. The director then decides which options have enough support to be included in the voting, before the voting can start.
    
    \item A voting consists of a number of options, each of which may be voted for. Wyvern members may vote for multiple options.

    \begin{item}
        There are two types of votings:
        \begin{enumerate}
            \item normal votings;
            \item unanimous votings.
        \end{enumerate}
    \end{item}

    \item If a decision must be made upon a money-related issue, the voting must be unanimous. Otherwise, the director decides whichever voting type is applicable before the vote is started.

    \begin{item}
        For a normal voting to succeed, the following conditions must be met:
        \begin{enumerate}
            \item at least 3 Wyvern members vote (non-abstaining);
            \item at least one Wyvern administrator participates (non-abstaining);
        \end{enumerate}
    \end{item}

    \begin{item}
        For a unanimous voting to succeed, the following conditions must be met:
        \begin{enumerate}
            \item at least 5 Wyvern members vote (non-abstaining);
            \item only a single option has more than zero votes.
        \end{enumerate}
    \end{item}

    \begin{item}
        In a successful voting, the most selected option is selected. In case of a tie, the administration team may (re)discuss the options, after which the director settles the vote by either:
        \begin{enumerate}
            \item taking an executive decision, thus deciding upon an option,
            \item restarting the vote, possibly with a modified set of options,
            \item postponing the vote, possibly indefinitely.
        \end{enumerate}
    \end{item}
    
    
\end{enumerate}

\subsection{Financial Responsibility}
In the Wyvern meeting of 2017-09-26, it was established that there should be shared financial responsibility in the unlikely, but possible, event of unexpected costs. This section formalizes that agreement. Above all, it is important that Wyvern members place good faith and trust each other to comply with their financial obligations subject to this section.

\begin{enumerate}
    
    \item At the end of each Wyvern term, a meeting must be held to discuss the continuation of the Wyvern project. In this meeting, new members ('potential members') may be present to join the Wyvern server.

    \item If during a meeting, a potential member agrees to join the Wyvern project, he or she becomes a Wyvern member and gains voting power and financial responsibility with respect to the contribution of the Wyvern server for the coming term.
    
    \item Wyvern members leaving the Wyvern project at the end of the term lose financial responsibility and voting power.
    
    \item Wyvern members agreeing to continue the Wyvern project for the next term during a meeting, he or she gains financial responsibility for the contribution of the next term.
    
    \begin{item}
         In case of unexpected costs resulting from the operation of the Wyvern server, services hosted by the Wyvern project, or by services hosted by members of the Wyvern community, the director must host an exceptional meeting to discuss the issue.
        
        \begin{enumerate}
            \item During such an exceptional meeting regarding unexpected financial damage, the origin must be determined and the extent of the damage must be quantified.

            \begin{item}
                The meeting must decide by a \emph{normal voting} one of the following two: 
                
                \begin{enumerate}
                    \item the financial damage has as basis the negligence of a single Wyvern member or a group thereof,
                    \item the financial damage has as basis the normal operation of Wyvern services.
                \end{enumerate}
            \end{item}
        
            \item In case that item (3.1.b.i) holds true, the financial responsibility and obligation to pay is placed upon the responsible Wyvern member or group thereof.
            
            \item In case that item (3.1.b.ii) holds true, the shared financial responsibility and obligation to pay is placed equally upon all Wyvern members.
            
            \item Financial responsibility includes the responsibility to pay the financial damage (or fraction of the financial damage) within the timespan and with the conditions to be decided upon by the director.
        \end{enumerate}

    \end{item}
\end{enumerate}

\newpage
