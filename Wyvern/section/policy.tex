\section{Policy}
\label{sec:policy}
This section roughly describes an overview of the server policies at a high level. To reiterate: these policies serve as a rough guideline and detailed overview of the exact policies are presented in a separate document. Whenever this section conflicts with that document, that document is leading. Note that the policies may change over time as the needs and desires of the Wyvern community change.

Most importantly, it is necessary users have \textit{trust} and \textit{respect} for one another. As this project is an ambitious and community-driven effort, it is in the best interest for the community as a whole that each user makes a wholehearted attempt to ensure everyone is content. This, in turn, means that in certain situations some concessions must be made. Whilst Wyvern should incorporate a democratic decision making process, each user must attempt to remain friendly to their co-users and resolve conflicts in a group effort.

\subsection{Fair Use}
As the Wyvern server is shared across users, the available resources are shared as well. This means that each user must put an effort to ensuring that their applications do not greatly overextend the server's resources. Some applications, such as game servers, will benefit multiple users, and thus may take up a large fraction of the allocated quota. Whilst a hard-limit is not placed on the available resources, users should use their best judgment and strive to avoid over-allocation. This does not mean that if a fraction of the server's resource capacity is unused, users should avoid allocating it. Using the server to its fullest extent is allowed, as long as other users are not limited.

\subsection{Democratic Management}
In effect, the Wyvern project is designed to be managed in a democratic fashion. If large changes are desired to be made, the Director should host community meetings in which Wyvern users may vote. The Director hosts these meetings to the best of their ability and ensure each user's wishes are taken into account. In general, for a decision to be taken, at least 50\% of the vote should be in favor. For topics related to finance, the vote should be unanimous.

\section{Technical Aspects}
\label{sec:technical}
Operating a community server such as Wyvern is only possible when certain technical decisions are taken with care.

\subsection{Operating System}
When selecting a proper operating system, it may be beneficial to chose either a RedHat-based (such as CentOS) or Debian-based (such as Ubuntu-Server) Linux distribution. The reason for this is the availability of on-line support for both operating system branches, as these are the most popular choices. For this reason, the Chief of Technology should advise the users on which option might be best suited.

\subsection{Users}
On Wyvern, each user should obtain a unique account which may be used to access the server through SSH. In addition, each user is allowed to store files and applications ins their home directories (\texttt{/home/<user>/}). In addition, users may share files by placing them in the \texttt{/home/shared/} folder. These directories may have an imposed disk-space quota to ensure no user is restricted in space. If specific platform applications or tool-chains (such as Java, Docker, etc.) need be installed with root-privileges, users should contact the Chief of Technology. Administrative users will each have a \textit{separate} administrator account which they may use to install such applications. However, this account should never be used to host personal applications.

\subsection{Security}
It is important that security plays a major role in the management of the server. For this reason, users may be required to log in to SSH through a cryptographic certificate instead of password-interactive authentication. In addition, requests for non-secure applications, such as outdated HTTP daemons and known applications with security vulnerabilities, may be declined.

\subsection{Shared Applications}
For certain applications such as game servers, it may be beneficial to allocate a separate user account so several users, hereafter referred to as \textit{application~owners}, may manage the application collaboratively. The fair use policy will be less strict for such applications.
